\chapter{数学公式与定理}

\section{符号示例}

Caligraphic letters: $\mathcal{A}$ 

Mathbb letters: $\mathbb{A}$

Mathfrak letters: $\mathfrak{A}$

Math Sans serif letters: $\mathsf{A}$

Math bold letters: $\mathbf{A}$

Math bold italic letters: $\mathbi{A}$

\section{公式示例}

% Thanks to github.com/YuanshengZhao/Garamond-Math
公式字体示例
\[
    \displaystyle \ointctrclockwise\mathcal{D}[x(t)]\sqrt{\frac{\displaystyle3\uppi^2-\sum_{q=0}^{\infty}(z+\hat L)^{q}\exp(\symrm{i}q^2 \hbar x)}{\displaystyle (\symsfup{Tr}\symbfcal{A})\left(\symbf\Lambda_{j_1j_2}^{i_1i_2}\Gamma_{i_1i_2}^{j_1j_2}\hookrightarrow\vec D\cdot \symbf P  \right)}}
    =\underbrace{\widetilde{\left\langle \frac{\notin \emptyset}{\varpi\alpha_{k\uparrow}}\middle\vert \frac{\partial_\mu T_{\mu\nu}}{2}\right\rangle}}_{\mathrm{K}_3\mathrm{Fe}(\mathrm{CN})_6} ,\forall z \in \mathbb{R}
\]

\begin{equation}\label{eq:dewitt}
    \int \mathrm{e}^{ax} \tanh {bx} \, \mathrm{d}x =
    \begin{dcases}
        \begin{multlined}
            \frac{\mathrm{e}^{(a+2b)x}}{(a + 2b)} \,
            {{}_2F_1} \left( 1 + \frac{a}{2b}, 1, 2+\frac{a}{2b}, -\mathrm{e}^{2bx} \right) \\
            - \frac{1}{a} \mathrm{e}^{ax} \, {{}_2F_1} \left( 1, \frac{a}{2b}, 1 + \frac{a}{2b}, -\mathrm{e}^{2bx} \right)
        \end{multlined}
        & a \ne b \\
    \frac{e^{ax} - 2 \tan^{-1}(\mathrm{e}^{ax})}{a} & a = b
  \end{dcases}
\end{equation}

你可以使用\verb|equation|环境插入公式,如\cref{eq:dewitt},代码如下:
\begin{lstlisting}
\begin{equation}
    \int \mathrm{e}^{ax} \tanh {bx} \, \mathrm{d}x =
    \begin{dcases}
        \begin{multlined}
            \frac{\mathrm{e}^{(a+2b)x}}{(a + 2b)} \,
            {{}_2F_1} \left( 1 + \frac{a}{2b}, 1, 2+\frac{a}{2b}, -\mathrm{e}^{2bx} \right) \\
            - \frac{1}{a} \mathrm{e}^{ax} \, {{}_2F_1} \left( 1, \frac{a}{2b}, 1 + \frac{a}{2b}, -\mathrm{e}^{2bx} \right)
        \end{multlined}
        & a \ne b \\
    \frac{e^{ax} - 2 \tan^{-1}(\mathrm{e}^{ax})}{a} & a = b
    \end{dcases}
\end{equation}
\end{lstlisting}

\section{定理环境}

% \begin{proof}
%     证明我是我
% \end{proof}

% \begin{definition}[他人]
%     定义他人即地狱
% \end{definition}

% 全部数学环境如下所示

% \begin{table}[htbp]
%     \caption{数学环境}
%     \label{tab:mathenv}
%     \begin{tabular}{cc}
%         \toprule
%         标签 & 名称 \\
%         \midrule
%         algorithm & 算法 \\
%         assumption & 假设 \\
%         axiom & 公理 \\
%         conclusion & 结论 \\
%         condition & 条件 \\
%         corollary & 推论 \\
%         definition & 定义 \\
%         example & 例 \\ 
%         lemma & 引理 \\
%         proof & 证明 \\
%         property & 性质 \\
%         proposition & 命题 \\
%         remark & 注解 \\
%         theorem & 定理 \\
%         \bottomrule
%     \end{tabular}
% \end{table}

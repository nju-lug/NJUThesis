\chapter{图片、表格与代码}

\section{图片}

插入一张图片,就像\cref{fig:arcaea}这样。

\begin{figure}[htbp]
    \centering
    \includegraphics[width=0.5\textwidth]{test.png}
    \caption{火热劲爆 Arcaea}
    \label{fig:arcaea}
\end{figure}
\zhlipsum[3][name=xiangyu]

\subsection{文字环绕图像}
% wrapfigure后面不能有空行
\begin{wrapfigure}{r}{0cm}
    \label{fig:arcaeaedge}
    \includegraphics[width=.15\textwidth]{test.png}
    \caption{闊靛緥婧愮偣}
\end{wrapfigure}
\zhlipsum[2][name=xiangyu]

% \begin{figure}[htbp]
%     \begin{subfigure}{.32\textwidth}
%         \centering
%         % \includegraphics[width=\textwidth]{njuemblem} 
%         \resizebox{\textwidth}{!}{\input{njuemblem.tikz}} 
%         \caption{logo1}
%     \end{subfigure}
%     \begin{subfigure}{.32\textwidth}
%         \centering
%         % \includegraphics[width=\textwidth]{njuemblem}  
%         \resizebox{\textwidth}{!}{\input{njuemblem.tikz}} 
%         \caption{logo2}
%     \end{subfigure}
%     \begin{subfigure}{.32\textwidth}
%         \centering
%         % \includegraphics[width=\textwidth]{njuemblem}  
%         \resizebox{\textwidth}{!}{\input{njuemblem.tikz}} 
%         \caption{logo3}
%     \end{subfigure}
%     \caption{njuemblems}
% \end{figure}
    
\section{表格}

你可以使用\lstinline|table|环境插入标准三线表,如\cref{tab:testtab}所示

\begin{table}[htbp]
    \centering
    \caption{经过测试的环境}
    \label{tab:testtab}
    \begin{tabular}{ccc}
        \toprule
        OS & TeX & 测试情况 \\
        \midrule
        Windows 10 & TeXLive 2021 & √ \\
        Windows 10 & MiKTeX & √ \\
        Windows 10 & TeXLive 2020 & ×\footnote{cleveref在引用章节时不能正常工作}  \\
        Ubuntu 20.04 & TeXLive 2021 & √ \\
        南大TeX & Overleaf & √ \\
        \bottomrule
    \end{tabular}
\end{table}

\section{代码}

\subsection{行内代码}
The new command pretty-prints the code. The exclamation marks delimit
the code and can be replaced by any character not in the code;
\verb$var i:integer;$ gives the same result.

使用\lstinline!\lstinline|\textit{<Your code>}|!,只要使用在代码中未出现的符号将代码包括在内即可。

使用 {algorithm} 和 {algorithmic} 的算法示例\footnote{两者均位于 {algoritms} 下}。

\begin{algorithm}[htbp]
    \caption{Temp}
    \begin{algorithmic}
        \IF{$i\leq0$}
        \STATE $i\gets1$
        \ELSE\IF{$i\geq0$}
        \STATE $i\gets0$
        \ENDIF\ENDIF
    \end{algorithmic}
\end{algorithm}

代码示例。

\begin{lstlisting}[language=python]
print(helloworld)
\end{lstlisting}
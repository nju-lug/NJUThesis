\chapter{图片、表格与代码}

\section{图片}

插入一张图片,就像\cref{fig:arcaea}这样。

\begin{figure}[htbp]
    \centering
    \includegraphics[width=0.5\textwidth]{test.png}
    \caption{火热劲爆 Arcaea}
    \label{fig:arcaea}
\end{figure}
\zhlipsum[3][name=aspirin]

\subsection{文字环绕图像}
% wrapfigure后面不能有空行
\begin{wrapfigure}{r}{0cm}
    \label{fig:arcaeaedge}
    \includegraphics[width=.15\textwidth]{test.png}
    \caption{闊靛緥婧愮偣}
\end{wrapfigure}
\zhlipsum[2][name=simp]

\subsection{多个图像}
\begin{figure}[htbp]
    \centering
    \begin{subfigure}{.32\textwidth}
        \centering
        \includegraphics[width=\textwidth]{test.png}
        \caption{闊靛緥婧愮偣}
    \end{subfigure}
    \begin{subfigure}{.32\textwidth}
        \centering
        \includegraphics[width=\textwidth]{test.png}
        \caption{Arcaea}
    \end{subfigure}

    \begin{subfigure}{.32\textwidth}
        \centering
        \includegraphics[width=\textwidth]{test.png}
        \caption{闊虫父}
    \end{subfigure}
    \caption{新概念空间立体节奏游戏}
\end{figure}
    
\section{表格}

你可以使用 |table| 环境插入标准三线表,如\cref{tab:testtab}所示

\begin{table}[htbp]
    \centering
    \caption{经过测试的环境}
    \label{tab:testtab}
    \begin{tabular}{ccc}
        \toprule
        OS & TeX & 测试情况 \\
        \midrule
        Windows 10 & TeXLive 2021 & √ \\
        Windows 10 & MiKTeX & √ \\
        Windows 10 & TeXLive 2020 & ×\footnote{cleveref在引用章节时不能正常工作}  \\
        Ubuntu 20.04 & TeXLive 2021 & √ \\
        南大TeX & Overleaf & √ \\
        \bottomrule
    \end{tabular}
\end{table}

\section{代码}

\subsection{行内代码}

一共有三种方法可以生成行内代码:
\begin{itemize}
    \item 使用 \texttt{\textbackslash verb} 抄录环境:\verb!<Your code>!
    \item 使用 \texttt{\textbackslash lstinline} 行内代码环境:\lstinline[basicstyle=\tt]+<Your code>+
    \item 使用简写的行内代码环境:|<Your code>|
\end{itemize}
在前两者中,只要使用在代码中未出现的符号将代码包括在内即可。

\subsection{代码块}

代码示例。在构建 \textsf{njuvisual} 宏包时,使用 inkscape 程序从矢量图导出的 \textsf{tikz} 曲线代码并不美观,需要缩进、对齐,因而拜托鄢老师编写了如下的 Python 脚本。
\begin{lstlisting}[language=Python,morekeywords={startswith,endswith,split,strip,join,find,append,replace}]
def deal_tuple(s):
if s.startswith("("):
    # s be like (x,y)
    return "( {0:>9}, {1:>9} )".format(*s[1:-1].split(","))
else:
    return s


indent = len(".. controls (    0.0000,    0.0000 )") - 1


def deal_line(s):
    s = " ".join(map(deal_tuple, s.split()))
    if s.find(")") != -1:
        s = " " * (indent - s.find(")")) + s
    return s


try:
    while True:
        s = input().strip()
        if s.startswith("\path"):
            l = []
            while not s.endswith(";"):
                l.append(s)  # use this instead of += for speedup
                s = input().strip()
            l.append(s[:-1])  # remove trailing ;
            # NOTE: all strings in l should be newline-free
            head, rp, body = " ".join(l).replace(".. controls", "\n.. controls").replace(
                "--", "\n--").replace(")(", ")\n(").replace("cycle(", "cycle\n(").partition("]")

            # force the program to keep newlines added manually
            result = "\n".join(map(deal_line, body.splitlines()))

            print(head, rp, "\n", result, ";", sep="")
        else:
            print(s)
except EOFError as e:
    pass
\end{lstlisting}

\subsection{算法块}

使用 {algorithm} 和 {algorithmic} 的算法示例\footnote{两者均位于 \textsf{algorithms} 下}。

\begin{algorithm}[htbp]
    \caption{Lanczos Algorithm}
    \begin{algorithmic}
        \STATE - Choose an initial state $\vec{b}_0$ and a maximum iteration number $m$
        \FOR{$n=0,\ldots m$}
        \STATE - Construct the space $K = \spn\{\vec{b}_0 , \ldots \vec{b}_n \}$
        \STATE - Obtain $\vec{C} = \hat{H} \vec{b}_n$
        \IF{$\vec{C} \in K$}
        \STATE Set $\vec{b}_{n+1}$ to a random vector orthogonal to $K$
        \ELSE
        \STATE Orthogonalize $\vec{C}$ against $K$, yielding $\vec{b}_{n+1}$
        \ENDIF\ENDFOR
    \end{algorithmic}
\end{algorithm}

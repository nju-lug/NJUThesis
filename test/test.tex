%%%%%%%%%%%%%%%%%%%%%%%%%%%%%%%%%%%%%%%%%%%%%%%%%%
% 本文档用于测试 njuthesis 是否正常工作
% 其内容不具有任何参考意义
%%%%%%%%%%%%%%%%%%%%%%%%%%%%%%%%%%%%%%%%%%%%%%%%%%

\ifdefined\TEST\else
  % 自行编译前需反注释以下任意命令
  % \documentclass{njuthesis}
  % \documentclass[oneside]{njuthesis}
  % \documentclass[cjk-font=noto,latin-font=gyre]{njuthesis}
  % \documentclass[cjk-font=source,latin-font=gyre]{njuthesis}
  % \documentclass[cjk-font=founder,latin-font=gyre]{njuthesis}
  % \documentclass[decl-page]{njuthesis}
  % \documentclass[draft]{njuthesis}
  % \documentclass[type=design]{njuthesis}
  \documentclass[degree=mg]{njuthesis}
  % \documentclass[degree=mf,oneside,decl-page]{njuthesis}
  % \documentclass[degree=phd,draft]{njuthesis}
  % \documentclass[degree=phd,nlcover]{njuthesis}
\fi

\njusetup {
    info = {
        title = {一种使用南京大学 \hologo{LaTeX} 模板\\编写毕业论文的经验性方法},
        title* = {An Empirical Way of Composing Thesis with NJU \hologo{LaTeX} Template},
        keywords = {我,就是,充数的,关键词},
        keywords* = {Dummy,Keywords,Here,{it is}},
        grade = {2018},
        % student-id = {DZ18114514},
        student-id = {189114514},
        author = {周煜华},
        author* = {Zhou~Yuhua},
        department = {拉太赫科学与技术学院},
        department* = {School~of~\hologo{LaTeX}},
        major = {拉太赫语言学},
        major* = {\hologo{LaTeX}~Linguistics},
        field = {拉太赫语言在现当代的使用},
        field* = {Contemporary~Usage~of~the~\hologo{LaTeX}~Language},
        supervisor = {李成殿,教授},
        supervisor* = {Professor Li Chengdian},
        % supervisor-ii = {孙赫弥,副教授},
        % supervisor-ii* = {Associate~professor~Sun~Hemi},
        % submit-date = {2021-08-10},
        % defend-date = {2021-09-19},
        chairman = {张晓山~教授},
        reviewer = {王瑞希~教授,郭德纲~副教授,华芈库~教授,戴菲菲~教授},
        clc = {0175.2},
        secret-level = {限制},
        udc = {004.72},
        supervisor-contact = {拉太赫科学与技术学院~枝江市结丘路~19~号~114514},
      },
    bib = {
        % style = author-year,
        resource = {test.bib},
        option = {doi=false, isbn=false, url=false, eprint=false}
      },
    style = {
        emblem-img = {wug},
        name-img = {name},
        abstract-in-toc = false,
        toc-in-toc = false,
        header/format = \small\heiti,
        header = {
          {OR}{\thepage},{EL}{\thepage},
          {ER}{\leftmark},{OL}{\rightmark}},
        % header = {},
        footer = {},
      }
  }

% \addbibresource{test.bib}

\ctexset{
  contentsname   = { 目\hspace{2em}次 },
%   listfigurename = { 插图清单 },
%   listtablename  = { 表格清单 }
}

% \usepackage{multirow,tabularray,wrapfig,subcaption}
% \usepackage{listings,algorithm,algorithmic}
% \usepackage{siunitx,physics,chemfig}
% \usepackage[version=4]{mhchem}
\usepackage{hologo,blindtext,zhlipsum}

% \setmonofont{cmun}[
%   Extension      = .otf,
%   UprightFont    = *btl,
%   BoldFont       = *tb,
%   ItalicFont     = *bto,
%   BoldItalicFont = *tx,
%   HyphenChar     = None]

% \lstdefinestyle{njulisting}
%   {
%     basewidth    = 0.5 em,
%     lineskip     = 3 pt,
%     basicstyle   = \tiny\ttfamily,
%     keywordstyle = \bfseries\ttfamily\color{njuviolet},
%     commentstyle = \itshape\ttfamily\color{gray},
%     stringstyle  = \color{njumagenta},
%     numbers      = left,
%     captionpos   = t,
%     breaklines   = true,
%     xleftmargin  = 2 em,
%     xrightmargin = 2 em
%   }
% \lstset{
%     style        = njulisting,
%     flexiblecolumns
%   }
% % \lstMakeShortInline[
% %   style=njulisting,
% %   basicstyle=\normalsize\tt,
% %   columns=fixed]|

% \setchemfig{
%   atom sep=14.4pt,
%   double bond sep=2.6pt,
%   bond style={line width=0.6pt},
%   cram width=2.0pt,
%   bond offset=1.6pt
% }
% \renewcommand*\printatom[1]{\small\ensuremath{\mathsf{#1}}}

% https://tex.stackexchange.com/questions/33264/span-as-a-math-operator
\DeclareMathOperator{\spn}{span}
\renewcommand{\vec}[1]{\mathbf{#1}}
% \RenewDocumentCommand\vec{m}{\mathbf{#1}}
\NewDocumentCommand\mathbi{m}{\textbf{\em #1}}

\begin{document}

% \frontmatter
\maketitle

% \begin{preface}
%   感慨万千,一时语塞。实在不知道写啥好,表演个朗诵课文吧。

%   \zhlipsum[108-110][name=zhufu]

%   \vspace{1cm}
%   \begin{flushright}
%     某人\\
%     2021年初秋于扬州
%   \end{flushright}
% \end{preface}

\begin{abstract}
  \zhlipsum[76-90][name=zhufu]
\end{abstract}

\begin{abstract*}
  \blindtext
\end{abstract*}

\raggedbottom
% \flushbottom

\tableofcontents
\listoffigures
\listoftables

\mainmatter

\begin{notation}[10cm]
  \item[DMRG] 密度矩阵重整化群 (Density-Matrix Reformation-Group)
  \item[DFT] 密度泛函理论 (Density functional theory)
  \item[PCR] 聚合酶链式反应 (Polymerase Chain Reaction)
  \item[HTTP] 超文本传输协议 (Hypertext Transfer Protocol)
  \item[CSS] 层叠样式表 (Cascading Style Sheets)
  \item[GUI] 图形用户界面 (Graphical User Interface)
  \item[CLI] 命令行界面 (Command-Line Interface)
  \item[IPO] 首次公开募股 (Initial Public Offering)
  \item[CDO] 抵押债务证券 (Collateralized Debt Obligations)
  \item[IPR] 国内称为贷款基础利率,美国、香港等地则称为最优惠贷款利率 (Loan Prime Rate)
  \item[$k_\mathrm{B}$] 玻尔兹曼常数
  \item[$h$] 普朗克常数
  \item[$\hbar$] 约化普朗克常数
  \item[$c$] 光速
  \item[$x$] 坐标
  \item[$p$] 动量
  \item[$t$] 时间
  \item[$\mu$] 约化质量
  \item[$I$] 转动惯量
  \item[呜呜呜] 特别长会怎样呢哈哈哈哈哈哈哈哈哈哈哈哈哈哈哈哈哈哈哈哈哈哈哈哈哈哈哈哈哈哈哈哈哈哈哈哈哈哈哈哈哈哈哈哈哈哈哈哈哈哈哈哈哈哈
  \item[呜呜呜] 特别长会怎样呢哈哈哈哈哈哈哈哈哈哈哈哈哈哈哈哈哈哈哈哈哈哈哈哈哈哈哈哈哈哈哈哈哈哈哈哈哈哈哈哈哈哈哈哈哈哈哈哈哈哈哈哈哈哈
  \item[呜呜呜] 特别长会怎样呢哈哈哈哈哈哈哈哈哈哈哈哈哈哈哈哈哈哈哈哈哈哈哈哈哈哈哈哈哈哈哈哈哈哈哈哈哈哈哈哈哈哈哈哈哈哈哈哈哈哈哈哈哈哈
  \item[呜呜呜] 特别长会怎样呢哈哈哈哈哈哈哈哈哈哈哈哈哈哈哈哈哈哈哈哈哈哈哈哈哈哈哈哈哈哈哈哈哈哈哈哈哈哈哈哈哈哈哈哈哈哈哈哈哈哈哈哈哈哈
\end{notation}

\chapter[非常长的标题不好看]{如果标题非常非常非常非常非常非常非常非常非常非常非常非常非常非常非常非常非常非常长会怎样呢}
\label{ch:longtitle}
\section[还是用短点的吧]{如果标题非常非常非常非常非常非常非常非常非常非常非常非常非常非常非常非常非常非常长会怎样呢}
\cref{ch:longtitle}答案是使用可选参数:\verb+\chapter[短描述]{完整的长标题}+

\zhlipsum[1-10][name=zhufu]

\subsection{测试}

\zhlipsum[11-20][name=zhufu]

\section{列表环境}

\begin{itemize}
    \item 测试测试测试测试测试测试测试测试测试测试测试测试测试测试测试测试测试测试测试测试测试测试测试测试测试测试测试测试测试测试测试测试测试测试测试测试
    \item 测试测试测试测试测试测试测试测试测试测试测试测试测试测试测试测试测试测试测试测试测试测试测试测试测试测试测试测试测试测试测试测试测试测试测试测试
    \item 测试测试测试测试测试测试测试测试测试测试测试测试测试测试测试测试测试测试测试测试测试测试测试测试测试测试测试测试测试测试测试测试测试测试测试测试
\end{itemize}

\begin{enumerate}
    \item 测试测试测试测试测试测试测试测试测试测试测试测试测试测试测试测试测试测试测试测试测试测试测试测试测试测试测试测试测试测试测试测试测试测试测试测试
    \item 测试测试测试测试测试测试测试测试测试测试测试测试测试测试测试测试测试测试测试测试测试测试测试测试测试测试测试测试测试测试测试测试测试测试测试测试
    \item 测试测试测试测试测试测试测试测试测试测试测试测试测试测试测试测试测试测试测试测试测试测试测试测试测试测试测试测试测试测试测试测试测试测试测试测试
\end{enumerate}

\begin{description}
    \item[测试测试] 测试测试测试测试测试测试测试测试测试测试测试测试测试测试测试测试测试测试测试测试测试测试测试测试测试测试测试测试测试测试测试测试测试测
    \item[测试测试] 测试测试测试测试测试测试测试测试测试测试测试测试测试测试测试测试测试测试测试测试测试测试测试测试测试测试测试测试测试测试测试测试测试测
    \item[测试测试] 测试测试测试测试测试测试测试测试测试测试测试测试测试测试测试测试测试测试测试测试测试测试测试测试测试测试测试测试测试测试测试测试测试测
\end{description}

\chapter{脚注}

测试脚注的编号效果%
\footnote{测试测试测试测试测试测试测试测试测试测试测试测试}%
\footnote{测试测试测试测试测试测试测试测试测试测试测试测试}%
\footnote{测试测试测试测试测试测试测试测试测试测试测试测试}%
\footnote{测试测试测试测试测试测试测试测试测试测试测试测试}%
\footnote{测试测试测试测试测试测试测试测试测试测试测试测试测试测试测试测试测试测试测试测试测试测试测试测试测试测试测试测试测试测试测试测试测试测试测试测试测试测试测试测试测试测试测试测试测试测试测试测试}%
\footnote{测试测试测试测试测试测试测试测试测试测试测试测试}%
\footnote{测试测试测试测试测试测试测试测试测试测试测试测试}%
\footnote{测试测试测试测试测试测试测试测试测试测试测试测试}%
\footnote{测试测试测试测试测试测试测试测试测试测试测试测试}%
\footnote{测试测试测试测试测试测试测试测试测试测试测试测试}%
\footnote{测试测试测试测试测试测试测试测试测试测试测试测试}%
\footnote{测试测试测试测试测试测试测试测试测试测试测试测试}%
\footnote{测试测试测试测试测试测试测试测试测试测试测试测试}%
\footnote{测试测试测试测试测试测试测试测试测试测试测试测试}%
\footnote{测试测试测试测试测试测试测试测试测试测试测试测试}%
\footnote{测试测试测试测试测试测试测试测试测试测试测试测试}%
\footnote{测试测试测试测试测试测试测试测试测试测试测试测试}%

\chapter{图片、表格与代码}

\section{图片}

插入一张图片,就像\cref{fig:arcaea}这样。

\begin{figure}[htbp]
    \centering
    \includegraphics[width=0.5\textwidth]{test.png}
    \caption{火热劲爆 Arcaea}
    \label{fig:arcaea}
\end{figure}
\zhlipsum[3][name=aspirin]

% \subsection{文字环绕图像}
% % wrapfigure后面不能有空行
% \begin{wrapfigure}{r}{0cm}
%     \label{fig:arcaeaedge}
%     \includegraphics[width=.15\textwidth]{test.png}
%     \caption{闊靛緥婧愮偣}
% \end{wrapfigure}
% \zhlipsum[2][name=simp]

% \subsection{多个图像}
% \begin{figure}[htbp]
%     \centering
%     \begin{subfigure}{.32\textwidth}
%         \centering
%         \includegraphics[width=\textwidth]{test.png}
%         \caption{闊靛緥婧愮偣}
%     \end{subfigure}
%     \begin{subfigure}{.32\textwidth}
%         \centering
%         \includegraphics[width=\textwidth]{test.png}
%         \caption{Arcaea}
%     \end{subfigure}

%     \begin{subfigure}{.32\textwidth}
%         \centering
%         \includegraphics[width=\textwidth]{test.png}
%         \caption{闊虫父}
%     \end{subfigure}
%     \caption{新概念空间立体节奏游戏}
% \end{figure}

\section{表格}

你可以使用 |table| 环境插入标准三线表,如\cref{tab:testtab}所示。

\begin{table}[htbp]
    \centering
    \caption{经过测试的环境}
    \label{tab:testtab}
    \begin{tabular}{ccc}
        \toprule
        OS           & TeX           & 测试情况 \\
        \midrule
        Windows 10   & TeX Live 2021 & √        \\
        Windows 10   & MiKTeX        & √        \\
        Windows 10   & TeX Live 2020 & ×        \\
        Ubuntu 20.04 & TeX Live 2021 & √        \\
        南大 TeX     & Overleaf      & √        \\
        \bottomrule
    \end{tabular}
\end{table}

全部预定义的数学环境如\cref{tab:mathenv}所示。

\begin{table}[htbp]
    \centering
    \caption{数学环境}
    \label{tab:mathenv}
    \begin{tabular}{cc|cc}
        \toprule
        标签       & 名称 & 标签       & 名称 \\
        \midrule
        axiom      & 公理 & lemma      & 引理 \\
        corollary  & 推论 & proof      & 证明 \\
        definition & 定义 & theorem    & 定理 \\
        example    & 例   &            &      \\
        \bottomrule
    \end{tabular}
\end{table}

% \chapter{学科特殊环境}

\section{代码}
% the listings package

\subsection{行内代码}

一共有三种方法可以生成行内代码:
\begin{itemize}
    \item 使用 \texttt{\textbackslash verb} 抄录环境:\verb!<Your code>!
    \item 使用 \texttt{\textbackslash lstinline} 行内代码环境:\lstinline[basicstyle=\tt]+<Your code>+
    \item 使用简写的行内代码环境:|<Your code>|
\end{itemize}
在前两者中,只要使用在代码中未出现的符号将代码包括在内即可。

\subsection{代码块}

代码示例。在构建 \textsf{njuvisual} 宏包时,使用 inkscape 程序从矢量图导出的 \textsf{tikz} 曲线代码并不美观,需要缩进、对齐,因而拜托鄢老师编写了如下的 Python 脚本。
\begin{lstlisting}[language=Python,morekeywords={startswith,endswith,split,strip,join,find,append,replace}]
def deal_tuple(s):
if s.startswith("("):
    # s be like (x,y)
    return "( {0:>9}, {1:>9} )".format(*s[1:-1].split(","))
else:
    return s


indent = len(".. controls (    0.0000,    0.0000 )") - 1


def deal_line(s):
    s = " ".join(map(deal_tuple, s.split()))
    if s.find(")") != -1:
        s = " " * (indent - s.find(")")) + s
    return s


try:
    while True:
        s = input().strip()
        if s.startswith("\path"):
            l = []
            while not s.endswith(";"):
                l.append(s)  # use this instead of += for speedup
                s = input().strip()
            l.append(s[:-1])  # remove trailing ;
            # NOTE: all strings in l should be newline-free
            head, rp, body = " ".join(l).replace(".. controls", "\n.. controls").replace(
                "--", "\n--").replace(")(", ")\n(").replace("cycle(", "cycle\n(").partition("]")

            # force the program to keep newlines added manually
            result = "\n".join(map(deal_line, body.splitlines()))

            print(head, rp, "\n", result, ";", sep="")
        else:
            print(s)
except EOFError as e:
    pass
\end{lstlisting}

\section{算法块}

使用 {algorithm} 和 {algorithmic} 的算法示例\footnote{两者均位于 \textsf{algorithms} 下}。

\begin{algorithm}[htbp]
    \caption{Lanczos Algorithm}
    \begin{algorithmic}
        \STATE - Choose an initial state $\vec{b}_0$ and a maximum iteration number $m$
        \FOR{$n=0,\ldots m$}
        \STATE - Construct the space $K = \spn\{\vec{b}_0 , \ldots \vec{b}_n \}$
        \STATE - Obtain $\vec{C} = \hat{H} \vec{b}_n$
        \IF{$\vec{C} \in K$}
        \STATE Set $\vec{b}_{n+1}$ to a random vector orthogonal to $K$
        \ELSE
        \STATE Orthogonalize $\vec{C}$ against $K$, yielding $\vec{b}_{n+1}$
        \ENDIF\ENDFOR
    \end{algorithmic}
\end{algorithm}


\section{物理}
% the physics package

\begin{equation}
    \expval{A}{\Psi}
\end{equation}


\section{化学}

\subsection{化学式}

1972年,R. J. Fiela、E. Koros、R. Noyes等通过实验对BZ振荡反应作了解释。其主要思想
是:系统中存在着两个受溴离子浓度控制的过程A和B,当$[\ce{Br-}]$高于临界浓度$[\ce{Br-}]_\text{crit}$时发生
A过程,当$[\ce{Br-}]$低于$[\ce{Br-}]_\text{crit}$时发生B过程。也就是说$[\ce{Br-}]$起着开关作用,它控制着从A到
B过程,再由B到A过程的转变。在A过程,由于化学反应$[\ce{Br-}]$降低,当$[\ce{Br-}]$低于$[\ce{Br-}]_\text{crit}$
时,B过程发生。在B过程中,\ce{Br-}再生,$[\ce{Br-}]$增加,当$[\ce{Br-}]$再次高于$[\ce{Br-}]_\text{crit}$时,A过程再次
发生。这样系统就在A过程与B过程间往复振荡。下面用\ce{BrO3- - Ce^{3+} - MA - H2SO4}系统为例
加以说明。\cite{qiujinhengetal2010}

% 当$[\ce{Br-}]$足够高时,发生下列A过程:
% \begin{equation}
%     \ce{BrO3- + Br- + 2H+ ->[k_1] HBrO2 +HOBr}
% \end{equation}
% \begin{equation}
%     \ce{HBrO2 + Br- + H+ ->[k_2] 2HOBr}
% \end{equation}
% 其中第一步是速率控制步,当达到准定态时,有
% \begin{equation*}
%     \left[\ce{HBrO2}\right] =\frac{k_1}{k_2}\left[\ce{BrO3-}\right]\left[\ce{H+}\right]
% \end{equation*}

% 当$[\ce{Br-}]$低时,\ce{Ce^{3+}}被氧化发生下列B过程:
% \begin{equation}
%     \ce{BrO3- + HBrO2 + 2H+ ->[k_3] 2HBrO2 + H2O}
% \end{equation}
% \begin{equation}
%     \ce{BrO2 + Ce^{3+} + H+ ->[k_4] HBrO2 + Ce^{4+}}
% \end{equation}
% \begin{equation}
%     \ce{2HBrO2 ->[k_5] BrO3- + HOBr + H+}
% \end{equation}

% 反应(3)是速率控制步,经反应(3)、反应(4)将自催化产生\ce{HBrO2},达到准定态
% 时,有
% \begin{equation*}
%     [\ce{HBrO2}] \approx \frac{k_3}{2k_5}[\ce{BrO3-}][\ce{H+}]
% \end{equation*}

% 由反应(2)和反应(3)可以看出:\ce{Br}和\ce{BrO3-}是竞争\ce{HBrO2}的。当$k_2[\ce{Br-}]>k_3[\ce{BrO3-}]$时,自催化过程不可能发生。自催化是BZ振荡反应中必不可少的步骤,否则该振荡不
% 能发生。\ce{Br-}的临界浓度为
% \begin{equation*}
%     [\ce{Br-}]_\text{crit} =\frac{k_3}{k_2}[\ce{BrO3-}] =\num{5e-6}[\ce{BrO3-}]
% \end{equation*}
% \ce{Br-}的再生可通过下列过程实现:
% \begin{equation}
%     \ce{4Ce^{4+} + BrCH(COOH)2 +H2O + HOBr ->[k_6] Br- + 4Ce^{3+} + 3CO2 + 6H+}
% \end{equation}
% 该系统的总反应为
% \begin{equation}
%     \ce{2H+ + 2BrO3- + 2CH2(COOH)2 ->  BrCH(COOH)2 + 3CO2 + 4H2O}
% \end{equation}
% 振荡的控制物种是\ce{Br-}。


\subsection{结构式}

% \begin{figure}[H]
%   \centering
%   \chemfig{[:18]*5(-=(-CHO)-N-=)}
% \end{figure}

\schemestart
  \setchemfig{chemfig style=blue} \chemname[3ex]{\chemfig{[:-60]*6(-=)([:90]*6(=-EDG))}}{Diene}
  \hspace{10mm}
  \setchemfig{chemfig style=red} \chemname[3ex]{\chemfig{[:-60]*6(=)([::90]-EWG)}}{Dienophile}
  \hspace{10mm}
  \arrow(.mid east--.mid west)[,1.5]
  \hspace{10mm}
  \chemfig{[:-60]*6(=[,,,,blue]-[,,,,blue]--[,,,,red](<[,,,,red]{\color{red}E}|{\color{red}WG})-(<[,,,,blue]{\color{blue}EDG})-[,,,,blue])}
\schemestop

% pkg modiagram
% \begin{modiagram}[names,distance=5cm,names-style={draw=black}]
% \atom[C]{left}{
%   2s = {3.6;pair},
%   2p = {6.3;up,up},
%   label = {
%     2sleft = {2s},
%     2pyleft = {2p}
%   }
% }
% \atom[O]{right}{
%   2s = {0;pair},
%   2p = {4.8;pair,up,up},
%   label = {
%     2sright = {2s},
%     2pyright = {2p}
%   }
% }
% \molecule[CO]{
%   2sMO = {0.4;pair,pair},
%   2pMO = {0.2/0.5,0.5/0.2;pair,pair,pair},
% }
% \node[right] at (2sigma.45) {$3\sigma$};
% \node[right] at (2sigma*.-45) {$4\sigma$};
% \node[right] at (2piz.45) { $1\pi$};
% \node[right] at (2psigma.45) {$5\sigma$};
% \node[right] at (2piz*.45) {$2\pi$};
% \node[right] at (2psigma*.45) {$6\sigma$};
% \EnergyAxis[title=Energy]
% \end{modiagram}

\chapter{数学公式与定理}

\section{公式示例}
\begin{equation}\label{eq:dewitt}
    \int  e^{ax} \tanh bx\ dx = 
    \begin{cases}
    \displaystyle{ \frac{ e^{(a+2b)x}}{(a+2b)} 
    {_2F_1}\left[ 1+\frac{a}{2b},1,2+\frac{a}{2b}, -e^{2bx}\right] }& \\
    \displaystyle{
    \hspace{1cm}-\frac{1}{a}e^{ax}{_2F_1}\left[ 1, \frac{a}{2b},1+\frac{a}{2b}, -e^{2bx}\right]
    }
     & a\ne b \\
    \displaystyle{\frac{e^{ax}-2\tan^{-1}[e^{ax}]}{a} } & a = b
    \end{cases}
\end{equation}
    
你可以使用\lstinline|equation|环境插入公式,如\cref{eq:dewitt},代码如下:
\begin{lstlisting}[language=TeX]
\begin{equation}\label{eq:dewitt}
    \int  e^{ax} \tanh bx\ dx = 
    \begin{cases}
    \displaystyle{ \frac{ e^{(a+2b)x}}{(a+2b)} 
    {_2F_1}\left[ 1+\frac{a}{2b},1,2+\frac{a}{2b}, -e^{2bx}\right] }& \\
    \displaystyle{
    \hspace{1cm}-\frac{1}{a}e^{ax}{_2F_1}\left[ 1, \frac{a}{2b},1+\frac{a}{2b}, -e^{2bx}\right]
    }
        & a\ne b \\
    \displaystyle{\frac{e^{ax}-2\tan^{-1}[e^{ax}]}{a} } & a = b
    \end{cases}
\end{equation}
\end{lstlisting}
\chapter{参考文献}
\label{chap:bibliography}

南京大学模板\cite{njuthesis}


\bgroup
  \sloppy
  \printbibliography
\egroup

\begin{acknowledgement}
  感谢\href{https://git.nju.edu.cn/nju-lug/lug-introduction}{NJU Linux User Group}。
\end{acknowledgement}

% \appendix

% \njuchapter{攻读博士学位期间研究成果及参与项目情况}

\njupaperlist{goossens93,segl03}
\njupaperlist*{goossens93,njuthesis}

\section*{攻读博士学位期间参与的科研课题}

\begin{enumerate}[label=\arabic*., labelindent=0em, leftmargin=*]
    \item 2020.1-2023.12, 国家自然科学基金, 12345678, 这是项目名称, 负责人: 老板1, 60 万.
    \item 2020.1-2022.12, 国家自然科学基金, 23456789, 这是项目名称, 负责人: 老板2, 25 万.
    \item 2020.1-2023.12, 国家自然科学基金, 34567890, 这是项目名称, 负责人: 老板3, 66 万.
\end{enumerate}

% \chapter[本科毕业论文格式规定]{南京大学本科毕业论文(设计)的撰写规范和装订要求(试行)}
\label{chap:standard}

本科毕业论文(设计)是本科教学中的重要环节,为规范本科毕业论文(设计)的工作,特制订毕业论文(设计)的撰写和装订要求,请同学们按照要求执行。如各院系已经制定了相应的规范,则按照院系的要求执行。

\section{毕业论文(设计)的撰写内容要求}
\subsection{论文题目}
论文题目应以简短、明确的词语恰当概括论文的核心内容,避免使用不常见的缩略词、缩写字。中文题目一般不宜超过40个字(含标点符号);外文题目一般不宜超过12个实词,最多不得超过180个字符(含标点符号)。
\subsection{摘要和关键词}
\subsubsection{中文摘要和中文关键词}
摘要内容应概括地反映出本论文的主要内容,主要说明本论文的研究目的、内容、方法、成果和结论。语言力求精练、准确,以300—600字为宜。

关键词是供检索用的主题词条。摘要与关键词应在同一页。关键词一般3—5个。

\subsubsection{英文摘要和英文关键词}
英文摘要和关键词内容与中文摘要和关键词相一致,其中英文摘要以约300个实词为宜。

\subsection{目录}
论文目录是论文的提纲,也是论文各章节组成部分的小标题。要求标题层次清晰,目录中的标题要与正文中的标题一致。

\subsection{主体}
主体部分一般从引言(绪论)开始,以结论或讨论结束,其中引言(绪论)应包括论文的研究目的、流程和方法等,论文研究领域的历史回顾,文献回溯,理论分析等内容,应独立成章,用足够的文字叙述;主体部分应从另页右页开始,每一章应另起页。

主体部分由于涉及的学科、选题、研究方法、结果表达方式等有很大的差异,不能作统一的规定。但是,必须实事求是,客观真切、准备完备、合乎逻辑、层次分明、简练可读。

\subsection{参考文献}
参考文献表是文中引用的有具体文字来源的文献集合,应置于正文后并另起页;所有被引用文献均要列入参考文献表中;引文采用著作-出版年制标注时,参考文献表应按著者字顺和出版年排序;

\subsection{相关的科研成果目录}
包括本科期间发表的与毕业论文(设计)相关的已发表论文或被鉴定的技术成果、发明专利等成果,应在成果目录中列出。此项非必需项。

\subsection{致谢}
谢辞应以简短的文字对课题研究与论文撰写过程中曾直接给予帮助的人员(例如指导教师、答疑教师及其他人员)表示自己的谢意,这不仅是一种礼貌,也是对他人劳动的尊重,是治学者应当遵循的学术规范。内容限一页。

\subsection{附录}
如果有不宜放在正文中的重要支撑材料,可编入毕业论文(设计)的附录中。包括某些重要的原始数据、详细数学推导、程序全文及其说明、复杂的图表、设计图纸等一系列需要补充提供的说明材料。附录的篇幅不宜太多,一般不超过正文。

\section{毕业论文(设计)的撰写规范要求}
\subsection{语言}
外语类专业的毕业论文(设计)应使用所学语种撰写,其他专业一般应使用中文撰写,具体按准出院系规定执行。非外语类专业如需用英文撰写,应于论文开题前由学生向准出院系提出申请,经院系批准后再进行撰写。以非中文完成的毕业论文(设计),应附上不少于2500字的中文详细摘要,作为该毕业论文(设计)的组成部分接受学术规范、答辩等所有审查评估。参加校级、省级本科生毕业论文(设计)评优的论文,如用英文撰写,必须同时提供中文版全文翻译(外语类专业除外)。

\subsection{字数}
除有特殊要求的专业外,毕业论文(设计)一般不少于15000字或相当信息量(包括图表)。如果院系有其他规定,按院系规定为准。

\subsection{字体和字号}
\begin{description}
    \item[论文题目] 三号宋体加粗
    \item[各部分标题] 四号黑体
    \item[中文摘要、关键词内容] 小四号楷体
    \item[英文摘要、关键词内容] 小四号新罗马体(Time New Roman)
    \item[目录标题] 三号宋体加粗
    \item[目录内容中章的标题] 四号黑体
    \item[目录中其他内容] 小四号宋体
    \item[正文] 小四号宋体(行距1.5倍)
    \item[参考文献标题] 四号黑体
    \item[参考文献内容] 小四号宋体
    \item[注释内容] 五号宋体
    \item[致谢、附录标题] 四号黑体
    \item[致谢、附录内容] 小四号宋体(行距1.5倍)
    \item[非正文部分的页码] 五号罗马数字(Ⅰ、Ⅱ……)
    \item[论文页码] 页脚居中、五号阿拉伯数字(新罗马体)连续编码
\end{description}

\subsection{关键词}
每个关键词之间用“;”分开,最后一个关键词不打标点符号。

\subsection{目录}
目录应另起一页,包括论文中的各级标题,按照“一……”、“(一)……”或“1……”、“1.1……”格式编写。

\subsection{各级标题}
正文各部分的标题应简明扼要,不使用标点符号。论文内文各大部分的标题用“一、二……(或1、2……)”,次级标题为“(一)、(二)……(或1.1、2.1……)”,三级标题用“1、2……(或1.1.1、2.1.1……)”,四级标题用“(1)、(2)……(或1.1.1.1、2.1.1.1……)”。不再使用五级以下标题。

\subsection{名词术语}
1、科学技术名词术语尽量采用全国自然科学名词审定委员会公布的规范词或国家标准中规定的名称,尚未统一规定或叫法有争议的名词术语,可采用惯用的名称。
2、特定含义的名词术语或新名词、以及使用外文缩写代替某一名词术语时,首次出现时应在括号内注明其含义。
3、外国人名一般采用英文原名,可不译成中文,英文人名按姓前名后的原则书写。一般很熟知的外国人名(如牛顿、爱因斯坦、达尔文、马克思等)可按通常标准译法写译名。

\subsection{图表的绘制}
表的题目在表的上方,图的题目在图的下方。图表的题目及内容的字体采用五号宋体。表中内容采用单倍行距。
图表的题目要简洁、加粗。图表的位置和题目皆要居中,与上下正文内容空一行。如果文中图表较多,建议采用章节+次序的办法编写,如第一章的第三个表为“表1-3”,第四章的第二个图为“图4-2”。通常表和图按各自顺序分开编号。

\subsection{注释}
毕业论文(设计)中有个别名词或情况需要解释时,可加注说明。注释采用脚注,每页独立编号,即每页都从1开始编码,编号用1,2,3……,文中编号用上标。
(十)参考文献
参考文献的著录应符合国家标准,参考文献的序号左顶格,并用数字加方括号表示,如“[1]”。每一条参考文献著录均以“.”结束。

\section{毕业论文(设计)装订要求}
论文在打印和印刷时,要求本科毕业论文封面使用浅绿色皮纹纸;本科毕业设计封面使用天蓝色皮纹纸;纸张的四周应留足空白边缘,以便于装订、复印和读者批注。每一面的上方(天头)和左侧(订口)应分别留边25mm以上间隙,下方(地角)和右侧(切口)应分别留边20mm以上间隙。所有在南大毕设系统填报的毕业论文过程记录(含开题报告、中期检查等)均可从系统导出打印,除需装订的内容其余在系统内做电子存档。
毕业论文(设计)应线装或胶装,并应按以下顺序装订:
\begin{enumerate}
    \item 封面
    \item 南京大学本科毕业设计(论文)诚信承诺书
    \item 本科毕业论文(设计)指导教师和评阅教师意见(注意填写的日期应在答辩前)
    \item 本科毕业论文(设计)答辩记录、成绩评定
    \item 中文摘要纸
    \item 英文摘要纸
    \item 目录
    \item 正文
    \item 参考文献
    \item 相关的科研成果目录(非必需项)
    \item 致谢
    \item 附录(非必需项)
\end{enumerate}


\end{document}
